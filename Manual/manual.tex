\documentclass[12pt]{amsart}
\usepackage{amsfonts, amsmath, latexsym, epsfig}
\usepackage{amssymb}
\usepackage{epsf}
\usepackage{url}
%\usepackage{tikz}


\newcommand{\sfA}{\ensuremath{\mathsf{A}}}
\newcommand{\RR}{\ensuremath{\mathbb{R}}}
\newcommand{\NN}{\ensuremath{\mathbb{N}}}
\newcommand{\QQ}{\ensuremath{\mathbb{Q}}}
\newcommand{\CC}{\ensuremath{\mathbb{C}}}
\newcommand{\ZZ}{\ensuremath{\mathbb{Z}}}
\newcommand{\TT}{\ensuremath{\mathbb{T}}}
\newcommand{\R}{\ensuremath{\mathbb{R}}}
\newcommand{\N}{\ensuremath{\mathbb{N}}}
\newcommand{\Q}{\ensuremath{\mathbb{Q}}}
\newcommand{\C}{\ensuremath{\mathbb{C}}}
\newcommand{\Z}{\ensuremath{\mathbb{Z}}}
\newcommand{\T}{\ensuremath{\mathbb{T}}}
\newtheorem{proposition}{Proposition}
\newtheorem{theorem}{Theorem}
\newtheorem{corollary}{Corollary}
\newtheorem{algorithm}{Algorithm}
\newtheorem{lemma}{Lemma}
\newtheorem{problem}{Problem}
\newtheorem{conjecture}{Conjecture}
\newtheorem{claim}{Claim}
\newtheorem{remark}{Remark}
\newtheorem{definition}{Definition}
\def\QuotS#1#2{\leavevmode\kern-.0em\raise.2ex\hbox{$#1$}\kern-.1em/\kern-.1em\lower.25ex\hbox{$#2$}}


\urlstyle{sf}

\DeclareMathOperator{\Aut}{Aut}
\DeclareMathOperator{\Sym}{Sym}
\DeclareMathOperator{\Isom}{Isom}
\DeclareMathOperator{\vertt}{vert}
\DeclareMathOperator{\conv}{conv}
\DeclareMathOperator{\SC}{SC}
\DeclareMathOperator{\SL}{SL}
\DeclareMathOperator{\GL}{GL}
\DeclareMathOperator{\PSL}{PSL}
\DeclareMathOperator{\Out}{Out}
\DeclareMathOperator{\Min}{Min}
\DeclareMathOperator{\Dom}{Dom}
\DeclareMathOperator{\cone}{cone}
\DeclareMathOperator{\Stab}{Stab}


\begin{document}

\author{Mathieu Dutour Sikiri\'c}
\address{Mathieu Dutour Sikiri\'c, Rudjer Boskovi\'c Institute, Bijenicka 54, 10000 Zagreb, Croatia}
\email{mdsikir@irb.hr}


\title{Short Manual of WWMIII}
\date{}

\maketitle


\section{Introduction}
WWMIII is a third generation wave model that solves the Wave Action Equation on an unstructured mesh.
It provides following interesting features:
\begin{enumerate}
\item It has many different source term formulations.
\item It is based on redistribution schemes and provide several advection
  schemes that can be both explicit and implicit.
\item It has coupling with many different geoscience models.
\end{enumerate}



\section{Compilation and running the model}

The WWM model can be run in serial mode, with MPI (Message Passing Interface) and/or OpenMP.








\section{Description of input file}
The input file is typically named {\tt wwminput.nml} and the program reads it as default by 
\begin{verbatim}
[mathieu@neptun ~]$ wwmadv
\end{verbatim}
But it is possible to give it other names, which are then used as
\begin{verbatim}
[mathieu@neptun ~]$ wwmadv wwminput_test.nml
\end{verbatim}
The input file is using a standard NAMELIST type:
\begin{enumerate}
\item Variables can be strings, logical, integers, floating point or list of those.
\item Logical tests are indicated as {\bf F/T}.
\item 
\item Optional items are set to a default value.
\item Some systems are very strict with the input that they accept. All strings should be delimited by '   '. Final comma at the end 
\end{enumerate}
We will see below all aspects of it and look at each section in turn.
The file {\tt wwminput.nml.ref} contains all the options and a short description of them



\subsection{Section {\tt PROC}}
This is the main section of the input file. It chooses the kind of grid used (spherical or not) the time stepping and other setting.
\begin{verbatim}
&PROC
 PROCNAME       = 'wwm_test'
 DIMMODE        = 2                  
 LSTEA          = F                  
 LQSTEA         = F                  
 LSPHE          = F                  
 LNAUTIN        = T
 LMONO_IN       = F                
 LMONO_OUT      = F                
 BEGTC          = '20071101.000000'
 DELTC          = 600              
 UNITC          = 'SEC'            
 ENDTC          = '20071102.000000'
 DMIN           = 0.001            
/
\end{verbatim}
The meaning of those options is indicated below:
\begin{enumerate}
\item {\tt PROCNAME} (string): it is the name of the project simulation and is not used thereafter.
\item {\tt DIMMODE} (integer): dimension mode. Default is 2 for two-dimensional waves.
\item {\tt LSTEA} (logical): Whether ot use steady mode or not. Default value is true.
\item {\tt LQSTEA} (logical): Whether to use the quasi-steady mode or not. Default is false.
\item {\tt LSPHE} (logical): Whether to use spherical coordinates (lon/lat in degree) or plane coordinates (xy in meter)
\item {\tt LNAUTIN} (logical): Whether to use nautical convention for wind direction as input.
\item {\tt LMONO\_IN} (logical): If true the wave height used in the boundary condition is assumed to be monochromatic. If false it is assumed to be a significant wave height. Default is false.
\item {\tt LMONO\_OUT} (logical): If true the wave height in output is assumed to be monochromatic. If false it is a significant wave height. Default is false.
\item {\tt BEGTC} (string): The starting date of the simulation. No default.
\item {\tt DELTC} (float): The size of the time step. No default.
\item {\tt UNITC} (string): the unit of the time step. No default.
\item {\tt ENDTC} (string): the end date of the simulation. No default.
\item {\tt DMIN} (float): the minimum water depth.
\end{enumerate}
The nautical convention for direction is following: 0 degree is from the north and 90 degree is from the east.
The mathematical convention for direction is following: 0 degree is to the east and 90 degree is to the north.
Conversion is done via
\begin{equation*}
Dir_{naut} = 270 - Dir_{math}.
\end{equation*}
The conversion between monochromatic wave height and significant wave height is done via
\begin{equation*}
H_{significant} = \sqrt{2} H_{monochromatic}
\end{equation*}




\subsection{Section {\tt COUPL}}
This is the coupling section of the namelist file. It specifies how the WWM is coupled to other models (if it is).
\begin{verbatim}
&COUPL
 LCPL           = T                
 LROMS          = F                
 LTIMOR         = F                
 LSHYFEM        = F                
 RADFLAG        = 'LON'
 LETOT          = F                
 NLVT           = 10     
 DTCOUP         = 600.   
 IMET_DRY       = 1
/
\end{verbatim}
The meaning of the options is the following:
\begin{enumerate}
\item {\tt LCPL} (logical): Whether the model is coupled with another model. Default is false.
\item {\tt LROMS} (logical): Whether the model is coupled with the circulation model ROMS by pipes.
\item {\tt LTIMOR} (logical): Whether the model is coupled with the sediment model TIMOR by pipes
\item {\tt LSHYFEM} (logical): Whether the model is coupled with the circulation model SHYFEM by pipes.
\item {\tt RADFLAG} (string): specifies the kind of effect of waves on the circulation. Possible options:
  \begin{enumerate}
  \item {\tt "LON"}: The Longuet-Higgins formulation of barotropic coupling extended vertically.
  \item {\tt "XIA"}: The Xia formulation of baroclinic effects.
  \item {\tt "VOR"}: The vortex-force formulation of baroclinic effects.
  \end{enumerate}
\item {\tt LETOT} (logical): Whether the coupling effect is computed from the wave spectra (the right option but more expensive) or from the mean wave quantities.
\item {\tt NLVT} (integer): The number of vertical levels in SHYFEM.
\item {\tt DTCOUP} (float): the time coupling. It must be an integral multiple of the time stepping of WWM. Default is to set it equal to that value.
\item {\tt IMET\_DRY} (integer): scheme for computing derivative at the wet/dry interface
\end{enumerate}

The logical {\tt LCPL} determines whether the WWM is coupled or not.
If True, then it can be
\begin{itemize}
\item With SELFE when compiled with SELFE.
\item With ROMS (using pipes) if LROMS=T
\item With ROMS when compiled with the PGMCL library and ROMS
\item With TIMOR (using pipes) if LTIMOR=T
\item With SHYFEM (using pipes) if LSHYFEM=T
\end{itemize}


\subsection{Section {\tt GRID}}
This section describe the spectral and spatial grid that are used by the wave model.
The choice made there directly affects the running time of the model.

\begin{verbatim}
&GRID
 LCIRD          = T                 
 LSTAG          = F                 
 MINDIR         = 340.              
 MAXDIR         = 7.5               
 MDC            = 36                
 FRLOW          = 0.04              
 FRHIGH         = 1.                
 MSC            = 21                
 IGRIDTYPE      = 3                 
 FILEGRID       = 'hgrid.gr3'
 LSLOP          = F          
 SLMAX          = 0.2        
 LVAR1D         = F          
 LOPTSIG        = F          
/
\end{verbatim}
The meaning of the options is following:
\begin{enumerate}
\item {\tt LCIRCD} (logical): Whether the directional grid go over the full circle or not. Default is true.
\item {\tt LSTAG} (logical): Whether the directional grid is staggered or not. Default is False.
\item {\tt MINDIR} (float): minimal direction. It only makes sense if LCIRCD=T. Default is 0.
\item {\tt MAXDIR} (float): maximal direction. It only makes sense if LCIRCD=T. Default is 360.
\item {\tt MDC} (integer): number of directions.
\item {\tt FRLOW} (float): lowest frequency (in Hz)
\item {\tt FRHIGH} (float): highest frequency (in Hz)
\item {\tt MSC} (integer): number of frequency
\item {\tt IGRIDTYPE} (integer): index describing the file format of the grid used.
\item {\tt FILEGRID} (string): The filename of the grid file.
\item {\tt LSLOP} (logical): bottom slope limiter. Default is false.
\item {\tt SLMAX} (float): maximum slope.
\item {\tt LVAR1D} (logical): This is for 1-dimensional mode only.
\item {\tt LOPTSIG} (logical): use optimal frequency bins.
\item {\tt APPLY\_DXP\_CORR} (logical): If true we try to resolve the -180 / 180 problem in global grids. Default is false.
\item {\tt USE\_EXACT\_FORMULA\_SPHERICAL\_AREA} (logical): Use exact formula for computing spherical triangle area. Needed if the grid contains the North or South pole. Default is false.
\item {\tt LEXPORT\_GRID\_WW3} (logical): If true the grid of WWM is exported to a WaveWatch III readable format. Default is false.
\end{enumerate}
For real applications, one should work with {\tt LCIRCD=T}. The choice {\tt LCIRCD=F} makes sense for laboratory experiments when the wave directions are in a specific quadrant.

The frequencies of the model are put in an exponential grid {\tt FR(1:MSC)} with {\tt FR(1)=FRLOW}, {\tt FR(MSC)=FRHIGH}, and {\tt FR(i+1) = XFR FR(i)}. The value of {\tt XFR} should be around 1.1

The {\tt IGRIDTYPE} value has following possibilities:
\begin{enumerate}
\item {\tt IGRIDTYPE=1}: This is the XFN {\tt system.dat} file format.
\item {\tt IGRIDTYPE=2}: This is a special format for periodic grids, which are of interest for laboratory or idealized case simulations.
\item {\tt IGRIDTYPE=3}: This is the SELFE format. Filename is typically {\tt hgrid.gr3}
\item {\tt IGRIDTYPE=4}: This is an old WWM grid format that should no longer be used.
\item {\tt IGRIDTYPE=5}: This is a netcdf grid format. It is produced for example by history files of WWM.
\end{enumerate}




\subsection{Section {\tt INIT}}
This is to specify how the initial condition is set up
\begin{verbatim}
&INIT
 LHOTR          = F   
 LINID          = F   
 INITSTYLE      = 1   
/
\end{verbatim}
The meaning of the options is the following:
\begin{enumerate}
\item {\tt LHOTR} (logical): specifies whether we use hotfile. By default set to false.
\item {\tt LINID} (logical): It should be set to {\tt .NOT. LHOTR}.
\item {\tt INITSTYLE} (integer): it specifies the kind of initial case:
  \begin{enumerate}
  \item if {\tt INITSTYLE=1} then the spectra is computed from JONSWAP spectrum using the initial wind.
  \item if {\tt INITSTYLE=2 and IBOUNDFORMAT=3} then the spectra is computed from the WW3 forecasts according to the IFREMER format
  \item if {\tt INITSTYLE=3} then the spectrum is the one obtained from the {\tt fort.10003} file (works only in serial mode.
  \end{enumerate}
\end{enumerate}
The details of the hotfile initial are to be found in the {\tt HOTFILE} section of the input file.


TODO: The LINID option possibly needs to be eliminated.





\subsection{Section {\tt BOUC}}
The construction of the boundary condition used in the model. The complex nature of this section is to take into account whether the spectrum is
\begin{enumerate}
\item time dependent (realistic case typically) or not.
\item spatially dependent (realistic case typically) or not
\item In parametric form (that is dependent on a few parameters) or not.
\end{enumerate}
Other important thing to decide is the origin of the spectrum. If it is time spatially dependent and/or time dependent then it has to come from another data source. This can be IFREMER wave forecasts. This can also be netcdf wave boundary which may be parametric or not.

\begin{verbatim}
&BOUC
 LBCSE          = F  
 LBCWA          = F  
 LBCSP          = F  
 LINHOM         = F  
 LBSP1D         = F   
 LBSP2D         = F   
 LBINTER        = F   
 LINDSPRDEG     = T   
 LPARMDIR       = F   
 FILEWAVE       = 'wave.bcd'        
 BEGTC          = '20071101.000000' 
 DELTC          = 1                 
 UNITC          = 'HR'              
 ENDTC          = '20071102.000000'
 FILEBOUND      = 'wwmbnd.gr3'  
 IBOUNDFORMAT   = 1     
 WBHS           = 0.181         
 WBSS           = 3.            
 WBTP           = 1.50         
 WBDM           = 90.0         
 WBDSMS         = 1.           
 WBDS           = 10.          
 WBGAUSS        = 0.1          
 WBPKEN         = 3.3        
 NCDF_HS_NAME   = 'hs'       
 NCDF_DIR_NAME  = 'dir'      
 NCDF_SPR_NAME  = 'spr'      
 NCDF_FP_NAME   = 'fp'       
 NCDF_F02_NAME  = 't02'      
 NETCDF_OUT_PARAM = F   
 NETCDF_OUT_SPECTRA = F 
 NETCDF_OUT_FILE        
 USE_SINGLE_OUT = T  
 /
\end{verbatim}
The meaning of the options is the following:
\begin{enumerate}
\item {\tt LBCSE} (logical): If true the boundary condition is time dependent. Default is false.
\item {\tt LBCWA} (logical): If true the boundary condition is parametric. Default is false.
\item {\tt LBCSP} (logical): If true the (non-parametric) wave spectra is prescribed. Default is false.
\item {\tt LINHOM} (logical): If true the wave spectra is spatially non-homogeneous. Default is false.
\item {\tt LBSP1D} (logical): The input spectrum is 1-dimensional. Default is false.
\item {\tt LBSP2D} (logical): The input spectra is 2-dimensional. Default is false.
\item {\tt LBINTER} (logical): Time interpolation of spectrum between two times in which the spectrum is available. Default is true.
\item {\tt LINDSPRDEG} (logical): If 1-d wave spectra are read this flag defines whether the input for the directional spreading is in degrees (true) or exponent (false).
\item {\tt LPARMDIR} (logical): If LPARMDIR is true than directional spreading is read from WBDS and must be in exponential format at this time, only valid for 1d Spectra.
\item {\tt FILEWAVE} (logical): File containing the boundary spectra. Related entries to {\tt FILEWAVE} are:
  \begin{itemize}
  \item {\tt BEGTC} (string): beginning time of data availability in FILEWAVE
  \item {\tt DELTC} (float): unit of time step
  \item {\tt UNITC} (string): unit of time
  \item {\tt ENDTC} (string): end time of data availability in FILEWAVE
  \end{itemize}
  This concerns the option XXX.
\item {\tt FILEBOUND} (string): the file containing the description of the boundary condition. The format depends on the {\tt IGRIDTYPE}. Possible values:
  \begin{itemize}
  \item {\tt -1}: island boundary point
  \item {\tt 1}: exterior boundary point.
  \item {\tt 0}: not on boundary.
  \item {\tt 2}: active boundary point.
  \item {\tt 3}: Neumann boundary point.
  \end{itemize}
\item {\tt IBOUNDFORMAT} (integer): this specifies the nature of boundary data. Possible values:
  \begin{itemize}
  \item {\tt 1}: This is parametric wave boundary condition with WWM format. (Use FILEWAVE?)
  \item {\tt 2}: This is parametric wave boundary condition with FVCOM format. 
  \item {\tt 3}: This is boundary condition using WW§ output from IFREMER.
  \item {\tt 4}: This is boundary condition using netcdf output files.
  \end{itemize}
\item If {\tt LINHOM=F}, {\tt LBCWA=T} and {\tt LBCSE=T} then the spectra is determined by just a few parameters that are actually in the WWM input file:
  \begin{itemize}
  \item {\tt WBHS} (float): the significant wave height on the boundary.
  \item {\tt WBTP} (float): the wave period used.
  \item {\tt WBSS} (integer): the type of spectrum chosen on the boundary:
    \begin{itemize}
    \item $\pm 1$: Pierson-Moskowitz
    \item $\pm 2$: JONSWAP
    \item $\pm 3$: all in one bin
    \item $4$: Gauss
    \end{itemize}
    The sign determine whether the {\tt WBTP} is peak period (+) or mean period (-).
  \item {\tt WBDM} (float): the mean wave direction.
  \item {\tt WBDSMS} (float): Directional spreading value in degrees (1) or as exponent (2).
  \item {\tt WBDS} (float): directional spreading at the boundary.
  \item {\tt WBGAUSS} (float): factor for Gauss distribution if {\tt WBSS=1}.
  \item {\tt WBPKEN} (float):  Peak enhancement factor for Jonswap Spectra if {\tt WBSS=2}.
  \end{itemize}
\item When selecting {\tt IBOUNDFORMAT=3} 5 netcdf files are needed with the IFREMER. The relevant options with their default values are:
  \begin{itemize}
  \item {\tt NCDF\_HS\_NAME = 'hs'}
  \item {\tt NCDF\_DIR\_NAME  = 'dir'}
  \item {\tt NCDF\_SPR\_NAME  = 'spr'}
  \item {\tt NCDF\_FP\_NAME   = 'fp'}
  \item {\tt NCDF\_F02\_NAME  = 't02'}
  \end{itemize}
\item It is possible to output the boundary condition from an actual run and those boundary condition may later be used in another WWM run as forcing data. Relevant variables are:
  \begin{itemize}
  \item {\tt NETCDF\_OUT\_PARAM} (logical): whether to output the variable for parametric boundary condition-
  \item {\tt NETCDF\_OUT\_SPECTRA} (logical): whether to output the spectra for the boundary condition
  \item {\tt NETCDF\_OUT\_FILE} (string): the filename of the output.
  \item {\tt USE\_SINGLE\_OUT} (logical): use single precision in the output file.
  \end{itemize}
  
  

\end{enumerate}






\subsection{Section {\tt WIND}}
The wind section specifies how the wind, is provided to the WWM model. It is the most essential


\begin{verbatim}
&WIND
 LWINDFROMWWM   = F      
 LSEWD          = F      
 LSTWD          = T      
 LCWIN          = T      
 LWDIR          = T      
 WDIR           = 140.0  
 WVEL           = 10.0   
 CWINDX         = 30.0   
 CWINDY         = 0.0    
 BEGTC          = '20030101.000000'
 DELTC          = 60.0             
 UNITC          = 'MIN'            
 ENDTC          = '20030102.000000'
 LINTERWD       = T                
 FILEWIND       = 'wind.dat'       
 IWINDFORMAT    = 1                
 GRIB_FILE_TYPE = 1   
 MULTIPLE_IN   = T 
/
\end{verbatim}
The meaning of the options is the following:
\begin{enumerate}
\item {\tt LWINDFROMWWM} (logical): Whether wind should come from WWM. Default is True, but SELFE users may use False in which case the wind comes from SELFE.
\item {\tt LSEWD} (logical): time dependent wind file
\item {\tt LCWIN} (logical): Constant wind in space. Default is false.
\item If {\tt LCWIN=T} and {\tt LSEWD=T} then the wind is constant in time and space. Thus it can be describe by two numbers:
  \begin{itemize}
  \item If {\tt LWDIR=T} then those are {\tt CWINDX} and {\tt CWINDY}.
  \item if {\tt LWDIR=F} then those are {\tt WDIR} and {\tt WVEL}.
  \end{itemize}
\item Otherwise, the wind is obtained by the selection of {\tt IWINDFORMAT}. Possible values are:
  \begin{itemize}
  \item {\tt 1}: ASCII file {\tt FILEWAVE} with the relevant entry
    \begin{itemize}
    \item {\tt BEGTC} (string): Beginning of the file.
    \item {\tt DELTC} (float): the separation between different data in {\tt FILEWAVE}
    \item {\tt UNITC} (string): the unit of time separation
    \item {\tt ENDTC} (string): ending of time 
    \end{itemize}
  \item {\tt 2}: DWD grib file converted to netcdf. List of files is in {\tt FILEWAVE}
  \item {\tt 3}: NOAA CFRS grib netcdf files. List of files is in {\tt FILEWAVE}
  \item {\tt 4}: NOAA NARR grib netcdf files. List of files is in {\tt FILEWAVE}
  \item {\tt 5}: NetCDF CF file {\tt FILEWAVE} with wind field in Uwind / Vwind in a finite difference grid.
  \item {\tt 6}: NetCDF CF file {\tt FILEWAVE} with wind field in Uwind / Vwind with values at the model nodes.
  \item {\tt 7}: Grib files from meteorological models. List of files is in {\tt FILEWAVE}
  \end{itemize}
\item If {\tt IWINDFORMAT=7} then the {\tt GRIB\_FILE\_TYPE} has to be selected. Possible options are:
  \begin{itemize}
  \item {\tt 1}: ECMWF (IFS) forecasts.
  \item {\tt 2}: COSMO model outputs.
  \item {\tt 3}: DHMZ (ALADIN) forecasts.
  \end{itemize}
\item {\tt MULTIPLE\_IN} (logical): If true all MPI nodes are reading data at the same time. If false, only one node is reading and sending data to others via MPI sends/recvs. Default is false.
\item {\tt LINTERWD} (logical): If true interpolate between time steps. Default is true.
  
\end{enumerate}
TODO: Clear the LSTWD vs LSEWD issue.





\subsection{Section {\tt CURR}}
This is the section for currents in the model.

\begin{verbatim}
&CURR !NOT USED WITH SELFE
 LSECU          = F                  ! Time dependend currents
 LSTCU          = F                  ! Steady current
 LCCUR          = F                  ! Constant current
 CCURTX         = 0.0                ! current x-vec
 CCURTY         = 0.0                ! current y-vec
 BEGTC          = '20040901.030000'  ! Beginn time
 DELTC          = 1800               ! Time step
 UNITC          = 'SEC'              ! Unit
 ENDTC          = '20040904.180000'  ! End time
 LINTERCU       = F                  ! Interpolate linear within the wind input time step
 FILECUR        = 'current.dat'      ! Current file name; input file format: write(*,*) curtx; write(*,*) curty
 LERGINP        = F                  ! read timor file for input ... ergzus.bin
 CURFAC         = 1.000000
 ICURRFORMAT    = 1
 MULTIPLE_IN   = T                   ! If T then all process are reading data
                                     ! If F then only process 0 does it and
                                     ! then spread to other nodes.
/
\end{verbatim}
The meaning of the options is the following:
\begin{enumerate}
\item {\tt LSECU} (logical): If selected we have time dependent currents. Default is false.
\item {\tt LCCUR} (logical): If selected we have constant currents over the domain. Default is false.
\item If {\tt LSECU=T} and {\tt LCCUR=T} then the current is described by two numbers, i.e. {\tt CCURTX} and {\tt CCURTY}.
\item 
\end{enumerate}
TODO: clear the meaning with {\tt LSTCU}.


\subsection{Section {\tt WALV}}
\begin{verbatim}
&WALV !NOT USED WITH SELFE
 LSEWL          = F                  ! Time dependend currents
 BEGTC          = ' '                ! Begin time
 DELTC          = 1                  ! Time step
 UNITC          = 'HR'               ! Unit
 ENDTC          = ' '                ! End time
 LINTERWL       = F                  ! Interpolate linear within the wind input time step
 LSTWL          = T                  ! Steady water level
 LCWLV          = T                  ! Constant water level
 CWATLV         = 0.0                ! elevation of the water level [m]
 IWATLVFORMAT   = 1
 FILEWATL       = '_'                ! water level file name; input file format: write(*,*) eta
 MULTIPLE_IN   = T                   ! If T then all process are reading data
                                     ! If F then only process 0 does it and
                                     ! then spread to other nodes.
/
\end{verbatim}


\subsection{Section {\tt ENGS}}
\begin{verbatim}
&ENGS !SOURCE TERMS
 MESNL          = 1                  ! SNL4; MESNL = 0 , MESNL = 1, (Discrete Interaction approx.)
 MESIN          = 1                  ! Wind input: Ardhuin etal (1), Cycle 4 (2), Makin & Stam (3); Babanin et al. (4), Cycle 3 (5), no wind (0) (default = 1; all others are just for rearch and explicitly not supported at all!!!)
 IFRIC          = 1                  ! Formulation for atmospheric boundary layer, (IFRIC = 1 for MESIN = 1, IFRIC = 4 for MESIN=3); only used if MESIN .GT. 2
 MESBF          = 1                  ! Bottomg friction: JONSWAP(1) (Default at 1); 2 - Madsen
 FRICC          = 0.067              ! Cjon - Bottom friction coefficient (always positive)
 MESBR          = 1                  ! Shallow water wave breaking; 0: no; 1: BJ78 (Default at 1)
 ICRIT          = 1                  ! Wave breaking criterion: set as 1 - SWAN, 2 - Dingemans - default = 2
 ALPBJ          = 0.5                ! Dissipation proportionality coefficient (0.5 is default)
 BRHD           = 0.78               ! Wave breaking coefficient for Const. type wave breaking criterion; range: 0.6-0.83 (suggested 0.78)
 LMAXETOT       = T                  ! Limit shallow water wave height by wave breaking limiter (default=T)
 MESDS          = 1                  ! Formulation for the whitecapping source function; usually same as MESIN
 MESTR          = 1                  ! Formulation for the triad 3 wave interactions (MESTR = 0 (off), MESTR = 1 (Lumped Triad Approx. (LTA) like SWAN) (Default at 1)
 TRICO          = 0.1                ! proportionality const. (\alpha_EB); default is 0.1
 TRIRA          = 5.                 ! ratio of max. freq. considered in triads over mean freq.; 2.5 is suggested
 TRIURS         = 0.1                ! critical Ursell number; if Ursell # < TRIURS; triads are not computed
 LPRECOMPST4    = T                  ! Precompute the indices and weightings for the Ardhuin et al. formulation and save to disk, if it is .F. then the saved
                                     ! values are read from disk
/
\end{verbatim}



\subsection{Section {\tt NUMS}}
\begin{verbatim}
&NUMS ! Numerical methods
 ICOMP          = 0
    ! This parameter controls the way how the splitting is done and whether implicit or explicit schemes are used for spectral advection
    ! ICOMP = 0
    ! This means that all dimensions are integrated using explicit methods. Similar
    ! to WW3, actually the same schemes are available in WW3 4.1.
    ! ICOMP = 1
    ! This mean that advection in geographical space is done using implicit
    ! Methods, source terms and spectral space are still integrated as done in
    ! explicitly.
    ! ICOMP = 2
    ! This means that the advection is done using implicit methods and that the
    ! source terms are integrated semi-implicit using Patankar rules and linearized
    ! source terms as done in SWAN. Spectral part is still a fractional step
    ! ICOMP = 3
    ! We use implicit method for advection, refraction, freq. shifting and source terms.

 AMETHOD        = 1
                                     ! AMETHOD controls the different Methods in geographical space
                                     ! AMETHOD = 0
                                     ! No Advection in geo. Space
                                     ! AMETHOD = 1
                                     ! Explicit N-Scheme for ICOMP = 0 and Implicit N-Scheme for ICOMP > 0
                                     ! AMETHOD = 2
                                     ! PSI-Scheme for ICOMP = 0 and Implicit
                                     ! Crank-Nicholson N-Scheme for ICOMP > 0
                                     ! AMETHOD = 3
                                     ! LFPSI Scheme for ICOMP = 0 and Implicit two time level N2 scheme for ICOMP > 0

                                     ! AMETHOD = 4
                                     ! Like AMETHOD = 1 but using PETSc based on small matrices MNP**2. this can be efficient on small to medium scale cluster up to say 128 Nodes.

                                     ! AMETHOD = 5
                                     ! Like AMETHOD = 1 but using PETSc and assembling the full matrix and the source terms at once (MNP * MDC * MSC)**2. number of equations
                                     ! this is for large scale applications

                                     ! Remark for AMETHOD = 4 and 5. This methods are new and only tested on a few cases where the results look reasonable and do not depend on the number of CPU's which
                                     ! valdiates the correct implementation. The scaling performance is anticipated to be "quite poor" at this time. Mandy different consituents influence the parallel speedup.
                                     ! Please let me know all the information you have in order to improve and accelarate the developement of implicit parallel WWM-III.
                                     ! Have fun ... Aron and Thomas.

                                     ! AMETHOD = 6 
                                     ! Like AMETHOD = 1 but using our own solver and assembling the full matrix and the source terms at once (MNP * MDC * MSC)**2. number of equations
                                     ! this is for HPC's with larger number of CPU's 

                                     ! AMETHOD = 7 
                                     ! Like AMETHOD = 1 but using our own solver and assembling the full matrix and the source terms at once (MNP * MDC * MSC)**2. number of equations
                                     ! this is for HPC's with larger number of CPU's


 SMETHOD        = 1
                                     ! This switch controls the way the source terms are integrated. 0: no source terms;
                                     ! 1: splitting using RK-3 and SI for fast and slow modes 2: semi-implicit;
                                     ! 3: R-K3 (if ICOMP=0 or 1) - slow; 4: Dynamic Splitting (experimental)

 DMETHOD        = 2
                                     ! This switch controls the numerical method in directional space.
                                     ! DMETHOD = 0
                                     ! No advection in directional space
                                     ! DMETHOD = 1
                                     ! Crank-Nicholson (RTHETA = 0.5) or Euler Implicit scheme (RTHETA = 1.0)
                                     ! DMEHOD = 2
                                     ! Ultimate Quickest as in WW3 (usually best)
                                     ! DMETHOD = 3
                                     ! RK5-WENO
                                     ! DMETHOD = 4
                                     ! Explicit FVM Upwind scheme
                                     ! DMETHOD = 5
                                     ! Implicit FVM Upwind scheme

 RTHETA         = 0.5                ! Weighing factor for DMETHOD = 1, not really useful since Crank Nicholson integration can only be montotone for CFL .le. 2
 LITERSPLIT     = F                  ! T: double Strang split; F: simple split (more efficienct). Default: F

 LFILTERTH      = F
                                     ! LFILTERTH: use a CFL filter to limit the advection vel. In directional space. This is similar to WW3.
                                     ! Mostly not used. WWMII is always stable.
 MAXCFLTH       = 1.0                ! Max Cfl in Theta space; used only if LFILTERTH=T
 FMETHOD        = 1
                                     ! This switch controls the numerical method used in freq. space
                                     ! = 0
                                     ! No Advection in spectral space
                                     ! = 1
                                     ! Ultimate Quickest as in WW3 (best)
                                     ! = 2
                                     ! Explicit UPWIND 
                                     ! = 3
                                     ! Implicit UPWIND
 LFILTERSIG     = F                  ! Limit the advection velocitiy in freq. space (usually F)
 MAXCFLSIG      = 1.0                ! Max Cfl in freq. space; used only if LFILTERSIG=T
 LLIMT          = T                  ! Switch on/off Action limiter, Action limiter must mostly be turned on.

 MELIM          = 1
                                     ! Formulation for the action limiter
                                     ! MELIM = 1 (default)
                                     ! Limiter according to the WAM group (1988)
                                     ! MELIM = 2
                                     ! Limiter according to Hersbach Janssen (1999)
                                     ! For MESIN = 1 and MESDS = 1, which represents Cycle 3 formulation or Ardhuin, or other formulations except Cycle4, use MELIM = 1 and LIMFAK = 0.1
                                     ! For MESIN = 2 and MESDS = 2, which represents Cycle 4 formulation, use MELIM = 2 and LIMFAK = 0.6
                                     ! For WAM we need MELIM = 3
 LIMFAK         = 0.6                ! Proportionality coefficient for the action limiter MAX_DAC_DT = Limfak * Limiter; see notes above for value
 LDIFR          = T                  ! Use phase decoupled diffraction approximation according to Holthuijsen et al. (2003) (usually T; if crash, use F)
 IDIFFR         = 1                  ! Extended WAE account for higher order effects WAE becomes nonlinear; 1: Holthuijsen et al. ; 2: Liau et al. ; 3: Toledo et al. (in preparation)
 LCONV          = F                  ! Estimate convergence criterian and write disk (quasi-steady - qstea.out)
 LCFL           = F                  ! Write out CFL numbers; use F to save time
 NQSITER        = 1                  ! # of quasi-steady (Q-S) sub-divisions within each WWM time step (trial and errors)
 QSCONV1        = 0.98               ! Number of grid points [%/100] that have to fulfill abs. wave height criteria EPSH1
 QSCONV2        = 0.98               ! Number of grid points [%/100] that have to fulfill rel. wave height criteria EPSH2
 QSCONV3        = 0.98               ! Number of grid points [%/100] that have to fulfill sum. rel. wave action criteria EPSH3
 QSCONV4        = 0.98               ! Number of grid points [%/100] that have to fulfill rel. avg. wave steepness criteria EPSH4
 QSCONV5        = 0.98               ! Number of grid points [%/100] that have to fulfill avg. rel. wave period criteria EPSH5
 LEXPIMP        = F                  ! Use implicit schemes for freq. lower than given below by FREQEXP; used only if ICOMP=0
 FREQEXP        = 0.1                ! Minimum frequency for explicit schemes; only used if LEXPIMP=T and ICOMP=0
 EPSH1          = 0.01               ! Convergence criteria for rel. wave height ! EPSH1 < CONVK1 = REAL(ABS(HSOLD(IP)-HS2)/HS2)
 EPSH2          = 0.01               ! Convergence criteria for abs. wave height ! EPSH2 < CONVK2 = REAL(ABS(HS2-HSOLD(IP)))
 EPSH3          = 0.01               ! Convergence criteria for the rel. sum of wave action ! EPSH3 < CONVK3 = REAL(ABS(SUMACOLD(IP)-SUMAC)/SUMAC)
 EPSH4          = 0.01               ! Convergence criteria for the rel. avg. wave steepness criteria ! EPSH4 < CONVK4 = REAL(ABS(KHS2-KHSOLD(IP))/KHSOLD(IP))
 EPSH5          = 0.01               ! Convergence criteria for the rel. avg. waveperiod ! EPSH5 < REAL(ABS(TM02-TM02OLD(IP))/TM02OLD(IP))
 LVECTOR        = T                  ! Use optmized propagation routines for large high performance computers e.g. at least more than 128 CPU. Try LVECTOR=F first.
 IVECTOR        = 2                  ! USed if LVECTOR=T; Different flavours of communications
                                     ! LVECTOR = 1; same propagation style as if LVECTOR = F, this is for testing and development
                                     ! LVECTOR = 2; all spectral bins are propagated with the same time step and communications is done only once per sub-iteration
                                     ! LVECTOR = 3; all directions with the same freq. are propgated using the same time step the communications is done for each freq.
                                     ! LVECTOR = 4; 2 but for mixed open-mpi, code has to be compiled with -openmp
                                     ! LVECTOR = 5; 3 but for mixed open-mpi, code has to be compiled with -openmp
                                     ! LVECTOR = 6; same as 2 but highly optmizied with respect to memory usage, of course it is must less efficient than 2
                                     ! remarks: if you are using this routines be aware that the memory amount that is used is approx. for LVECTOR 1-5 arround
                                     ! 24 * MSC * MDC * MNP, so if you are trying this on 1 CPU you get a segmentation fault if your system has not enough memory or
                                     ! if your system is not properly configured it may results into the fact that your computer starts blocking since it try's to swap to disk
                                     ! The total amount of memoery used per CPU = 24 * MSC * MDC * MNP / No.CPU
 LADVTEST       = F                  ! for testing the advection schemes, testcase will be added soon
 LCHKCONV       = F                  ! needs to set to .true. for quasi-steady mode. in order to compute the QSCONVi criteria and check them
 DTMIN_DYN      =   1.00000000000000 ! min. time step for dynamic integration 
 NDYNITER       =         100,       ! max. iteration for dyn. scheme afterwards the limiter is applied in the last step 
 DTMIN_SIN      =   1.00000000000000 ! min. time steps for the full fractional step method, where each source term is integrated with it's own fractional step
 DTMIN_SNL4     =   1.00000000000000 !    
 DTMIN_SDS      =   1.00000000000000 !    
 DTMIN_SNL3     =   1.00000000000000 !   
 DTMIN_SBR      =  0.100000000000000 !    
 DTMIN_SBF      =   1.00000000000000 !    
 NDYNITER_SIN   =          10,       ! max. iterations for each source term in the fractional step approach. 
 NDYNITER_SNL4  =          10,       !
 NDYNITER_SDS   =          10,       !
 NDYNITER_SBR   =          10,       !
 NDYNITER_SNL3  =          10,       ! 
 NDYNITER_SBF   =          10,       !
 LZETA_SETUP    = F,                 ! T/F, whether to compute the wave setup.
 ZETA_METH      = 0,                 ! 0: use a simple conjugate gradient
                                     !    preconditioner
                                     ! 1: use PETSC
 SOLVERTHR       =  1.000000000000000E-5, ! Thr for the Block-Jacobi or Block-Gauss-Seider solver
 MAXITER =         100,                   ! Max. number of iterations 
 LNANINFCHK      = T,                     ! Check for NaN and INF
 LZETA_SETUP     = F,                     ! Compute wave setup 
 ZETA_METH       =           0,           ! Method for wave setup, Mathieu please explain!
 LSOURCESWAM     = T,                     ! Use ECMWF WAM formualtion for deep water physics 
 PMIN    =  0.000000000000000E+000,       ! Max. percentage of non-converged grid points 
 LSOURCESWWIII   = F,                     ! WW3 Ardhuin et al. sources, not working yet in this code use MESIN = 1 
 BLOCK_GAUSS_SEIDEL      = T,             ! Use the Gauss Seidel on each
                                          ! computer block. The result seems
                                          ! to be faster and use less memory
                                          ! But the results depend on the
                                          ! number of processors
 LNONL   = F                              ! Solve the nonlinear system using simpler algorithm (Patankar) 
 ASPAR_LOCAL_LEVEL =                      ! Aspar locality level
                                          ! 0 ASPAR_JAC(MSC,MDC,MNP),
                                          ! CAD_THE(MSC,MDC,MNP), CAS_SIG(MSC,MDC,MNP)
                                          ! are allocated
                                          ! 1: only ASPAR_JAC
                                          ! 2: only small local arrays.
 L_SOLVER_NORM = F                        ! Compute solver norm as termination
                                          ! check of jacobi-Gauss-Seidel
                                          ! solver
/
\end{verbatim}

\subsection{Section {\tt HISTORY}}
This is the code for doing the history output, that is variable defined all over the domain.
\begin{verbatim}
&HISTORY
 BEGTC          = '20110222.000000'
 DELTC          = 600          
 UNITC          = 'SEC'        
 ENDTC          = '20110226.000000'
 OUTSTYLE       = 'NC'       
 DEFINETC       = 86400      
 MULTIPLEOUT      = 0        
 USE_SINGLE_OUT  = T         
 PARAMWRITE      = T       
 GRIDWRITE       = T       
 IOBPD           = F       
 PRINTMMA       = F        
 FILEOUT        = 'misc.dat'
 HS             = F                  ! 1: significant wave height
 TM01           = F                  ! 2: mean period
 TM02           = F                  ! 3: zero-crossing mean period
 TM10           = F                  ! 4: 
 KLM            = F                  ! 5: mean wave number
 WLM            = F                  ! 6: mean wave length
 ETOTC          = F                  ! 7: Variable ETOTC
 ETOTS          = F                  ! 8: Variable ETOTS
 DM             = F                  ! 9: mean wave direction
 DSPR           = F                  !10: directional spreading
 TPPD           = F                  !11: discrete peak wave period
 CPPD           = F                  !12: discrete peak wave speed
 KPPD           = F                  !13: discrete peak wave number
 CGPD           = F                  !14: discrete peak group speed
 TPP            = F                  !15: peak period  
 CPP            = F                  !16: peak phase vel. 
 WNPP           = F                  !17: peak wave number
 CGPP           = F                  !18: peak group speed
 KPP            = F                  !19: peak wave number
 LPP            = F                  !20: peak wave length 
 PEAKD          = F                  !21: peak direction
 PEAKDSPR       = F                  !22: peak directional spreading
 DPEAK          = F                  !23: discrete peak direction 
 UBOT           = F                  !24: bottom exc. vel. 
 ORBITAL        = F                  !25: bottom orbital vel. 
 BOTEXPER       = F                  !26: bottom exc.  
 TMBOT          = F                  !27: bottom period 
 URSELL         = F                  !28: Ursell number
 UFRIC          = F                  !29: air friction velocity
 Z0             = F                  !30: air roughness length
 ALPHA_CH       = F                  !31: Charnoch coefficient for air
 WINDX          = F                  !32: Wind in X direction
 WINDY          = F                  !33: Wind in Y direction
 CD             = F                  !34: Drag coefficient
 CURRTX         = F                  !35: current in X direction
 CURRTY         = F                  !36: current in Y direction
 WATLEV         = F                  !37: water level
 WATLEVOLD      = F                  !38: water level at previous time step
 DEPDT          = F                  !39: change of water level in time
 DEP            = F                  !40: depth
 WINDMAG        = F                  !41: Wind magnitude
 TAUW           = F                  !42: surface stress from the wave
 TAUWX          = F                  !43: surface stress in X direction
 TAUWY          = F                  !44: surface stress in Y direction
 TAUHF          = F                  !45: high frequency surface stress
 TAUTOT         = F                  !46: total surface stress
 STOKESBOTTX    = F                  !47: Bottom Stokes drift in X direction
 STOKESBOTTY    = F                  !48: Bottom Stokes drift in X direction
 STOKESSURFX    = F                  !49: Surface Stokes drift in X direction
 STOKESSURFY    = F                  !50: Surface Stokes drift in X direction
 STOKESBAROX    = F                  !51: Barotropic Stokes drift in X direction
 STOKESBAROY    = F                  !52: Barotropic Stokes drift in Y direction
 RSXX           = F                  !53: RSXX potential of LH
 RSXY           = F                  !54: RSXY potential of LH
 RSYY           = F                  !55: RSYY potential of LH
 CFL1           = F                  !56: CFL number 1
 CFL2           = F                  !57: CFL number 2
 CFL3           = F                  !58: CFL number 3
 ZETA_SETUP     = F                  !59: wave induced setup.
/
\end{verbatim}
The meaning of the input file is the following:
\begin{enumerate}
\item {\tt BEGTC} (string): Beginning time of history output.
\item {\tt DELTC} (float): the time between history 
\item {\tt UNITC} (string): unit of output for the history.
\item {\tt ENDTC} (string): ending time of output.
\item {\tt OUTSTYLE} (string): type of output used. Possible values:
  \begin{itemize}
  \item {\tt ``NO''}: No history output is done.
  \item {\tt ``NC''}: netCDF style of data output.
  \item {\tt ``XFN''}: data output for the XFN plotting software.
  \item {\tt ``SHP''}: data output for the SHP software.
  \end{itemize}
\item {\tt PRINTMMA} (logical): print the minimum, maximum and average value of the variables that are in input. The output file is {\tt wwmstat\_0000}.
\item {\tt FILEOUT} (string): the filename of the output. If netCDF then it is edited to have {\tt .nc} as suffix.  
\end{enumerate}
Following options are only for netCDF:
\begin{enumerate}
\item {\tt DEFINETC} (float): time in second contained in the data output.
If set to a negative value then only one file is generated.
If set to a value of 86400 daily output files are created.
\item {\tt MULTIPLEOUT} (integer): if 0 then all MPI process write output into a single netCDF file. If not then each process does its own output into separate independent files.
\item {\tt USE\_SINGLE\_OUT} (logical): if true then the output is in single precision even if the model runs in double precision.
\item {\tt PARAMWRITE} (logical): write the physical parametrization used by the WWM model. This can avoid some errors. Default is true.
\item {\tt GRIDWRITE} (logical): write the model grid in the netCDF into the output file. In turn this can be used as input grid file for WWM in the case {\tt IGRIDTYPE=5}. Default is T.
\item {\tt IOBPD} (logical): write the {\tt IOBPD} variable in the output file.
\end{enumerate}
The name of the variable are mostly self-explanatory.





\subsection{Section {\tt STATION}}
This is the section for doing station output at predefined locations. Maximum number of output locations is 50.
\begin{verbatim}
&STATION
 BEGTC          = '20030101.000000' 
 DELTC          = 600           
 UNITC          = 'SEC'         
 ENDTC          = '20030102.000000' 
 DEFINETC       = 86400     
 OUTSTYLE       = 'NO'     
 MULTIPLEOUT      = 0       
 USE_SINGLE_OUT  = T       
 PARAMWRITE      = T     
 FILEOUT        = 'misc.dat'
 LOUTITER       = F
 IOUTS          = 8                  ! Number of output stations
 NOUTS          = 'P-1', 'P-2', 'P-3', 'P-4', 'P-5', 'P-6', 'P-7', 'P-8'! Name of output locations
 XOUTS          = 7280., 7257., 7324., 7319., 7302., 7270., 7958., 8663.! X-Coordinate of output locations
 YOUTS          = 19574., 18549., 17129., 15549., 14096., 12615., 8669., 5598.! Y-Coordinate of output locations
 CUTOFF         = 8*0.44             ! cutoff freq (Hz) for each station - consistent with buoys
 LSP1D          = T                  ! 1D spectral station output
 LSP2D          = F                  ! 2D spectral station output
 LSIGMAX        = T                  ! Adjust the cut-freq. for the output (e.g. consistent with buoy cut-off freq.)
! Below is selection for station specific variables. Default is F for all variables.
 AC             = F                  ! spectrum 
 WK             = F                  ! variable WK
 ACOUT_1D       = F                  ! variable ACOUT_1D
 ACOUT_2D       = F                  ! variable ACOUT_2D
! Below is selection for all variables. Default is F for all variables.
 HS             = F                  ! 1: significant wave height
 TM01           = F                  ! 2: mean period
 TM02           = F                  ! 3: zero-crossing mean period
 TM10           = F                  ! 4: 
 KLM            = F                  ! 5: mean wave number
 WLM            = F                  ! 6: mean wave length
 ETOTC          = F                  ! 7: Variable ETOTC
 ETOTS          = F                  ! 8: Variable ETOTS
 DM             = F                  ! 9: mean wave direction
 DSPR           = F                  !10: directional spreading
 TPPD           = F                  !11: discrete peak wave period
 CPPD           = F                  !12: discrete peak wave speed
 KPPD           = F                  !13: discrete peak wave number
 CGPD           = F                  !14: discrete peak group speed
 TPP            = F                  !15: peak period  
 CPP            = F                  !16: peak phase vel. 
 WNPP           = F                  !17: peak wave number
 CGPP           = F                  !18: peak group speed
 KPP            = F                  !19: peak wave number
 LPP            = F                  !20: peak wave length 
 PEAKD          = F                  !21: peak direction
 PEAKDSPR       = F                  !22: peak directional spreading
 DPEAK          = F                  !23: discrete peak direction 
 UBOT           = F                  !24: bottom exc. vel. 
 ORBITAL        = F                  !25: bottom orbital vel. 
 BOTEXPER       = F                  !26: bottom exc.  
 TMBOT          = F                  !27: bottom period 
 URSELL         = F                  !28: Ursell number
 UFRIC          = F                  !29: air friction velocity
 Z0             = F                  !30: air roughness length
 ALPHA_CH       = F                  !31: Charnoch coefficient for air
 WINDX          = F                  !32: Wind in X direction
 WINDY          = F                  !33: Wind in Y direction
 CD             = F                  !34: Drag coefficient
 CURRTX         = F                  !35: current in X direction
 CURRTY         = F                  !36: current in Y direction
 WATLEV         = F                  !37: water level
 WATLEVOLD      = F                  !38: water level at previous time step
 DEPDT          = F                  !39: change of water level in time
 DEP            = F                  !40: depth
 WINDMAG        = F                  !41: Windmagnitude
 TAUW           = F                  !42: surface stress from the wave
 TAUWX          = F                  !43: surface stress in X direction
 TAUWY          = F                  !44: surface stress in Y direction
 TAUHF          = F                  !45: high frequency surface stress
 TAUTOT         = F                  !46: total surface stress
 STOKESBOTTX    = F                  !47: Bottom Stokes drift in X direction
 STOKESBOTTY    = F                  !48: Bottom Stokes drift in X direction
 STOKESSURFX    = F                  !49: Surface Stokes drift in X direction
 STOKESSURFY    = F                  !50: Surface Stokes drift in X direction
 STOKESBAROX    = F                  !51: Barotropic Stokes drift in X direction
 STOKESBAROY    = F                  !52: Barotropic Stokes drift in Y direction
 RSXX           = F                  !53: RSXX potential of LH
 RSXY           = F                  !54: RSXY potential of LH
 RSYY           = F                  !55: RSYY potential of LH
 CFL1           = F                  !56: CFL number 1
 CFL2           = F                  !57: CFL number 2
 CFL3           = F                  !58: CFL number 3
 ZETA_SETUP     = F                  !59: wave induced setup.
/
\end{verbatim}
The measning of those options is the following:
\begin{enumerate}
\item {\tt BEGTC} (string): starting time of station output.
\item {\tt DELTC} (float): unit of time between station output.
\item {\tt UNITC} (string): unit of time between outputs.
\item {\tt ENDTC} (string): end time of station output.
\item {\tt OUTSTYLE} (string): type of output used. Possible values:
  \begin{itemize}
  \item {\tt ``NO''}: No history output is done.
  \item {\tt ``NC''}: netCDF style of data output.
  \item {\tt ``STE''}: ascii data output for text analysis.
  \end{itemize}
\end{enumerate}
Following options are only for netCDF:
\begin{enumerate}
\item {\tt DEFINETC} (float): time in second contained in the data output.
If set to a negative value then only one file is generated.
If set to a value of 86400 daily output files are created.
\item {\tt MULTIPLEOUT} (integer): if 0 then all MPI process write output into a single netCDF file. If not then each process does its own output into separate independent files.
\item {\tt USE\_SINGLE\_OUT} (logical): if true then the output is in single precision even if the model runs in double precision.
\item {\tt PARAMWRITE} (logical): write the physical parametrization used by the WWM model. This can avoid some errors. Default is true.
\end{enumerate}
The output data is then according to the options in a parallel way to section HISTORY.



\subsection{Section {\tt HOTFILE}}
This section specifies how the hotfile is written and read (but note that decision whether to use hotfile or not is done in the INIT section).

\begin{verbatim}
&HOTFILE
 LHOTF          = F
 BEGTC          = '20000101.000000' 
 DELTC          = 1     
 UNITC          = 'SEC' 
 ENDTC          = '20000101.100000' 
 LCYCLEHOT      = T
 HOTSTYLE_OUT   = 2
 MULTIPLEOUT    = 0
 FILEHOT_OUT    = 'hotfile_out.dat' 
 HOTSTYLE_IN    = 2 
 IHOTPOS_IN        = 1
 MULTIPLEIN     = 0 
 FILEHOT_IN     = 'hotfile_in.dat'
/
\end{verbatim}
The meaning of the options is the following:
\begin{enumerate}
\item {\tt LHOTF} (logical): Write hotfile. Default is false.
\item {\tt BEGTC} (string): Beginning time of writing hotfile.
\item {\tt DELTC} (float): separation between hotfile printing.
\item {\tt UNITC} (string): unit of the separation time.
\item {\tt ENDTC} (string): end time of writing hotfile.
\item {\tt LCYCLEHOT} (logical): It is valid only if {\tt HOTFILE\_OUT=T}. If selected then we have in the netCDF file only the last two entries. Otherwise, the last N states are present.
\item {\tt HOTSTYLE\_OUT} (integer): the kind of file format used for the hotfile output. Possible value:
  \begin{itemize}
  \item {\tt 1}: the file is binary.
  \item {\tt 2}: the file is netCDF
  \end{itemize}
\item {\tt MULTIPLEOUT} (integer): If equal to zero there is a single hotfile. If 1 then each MPI process creates its own hotfile.
\item {\tt FILEHOT\_OUT} (string): The name of the output file.
\item {\tt HOTSTYLE\_IN} (integer): the kind of file format used for hotfile input. See {\tt HOTSTYLE\_OUT} for possible values.
\item {\tt IHOTPOS\_IN} (integer): specific index of hotfile used (only for netCDF).
\item {\tt MULTIPLEIN} (integer): Whether we read from a single file or several.
\item {\tt FILEHOT\_IN} (string): the name of the hotfile used as input.
\end{enumerate}



\subsection{Section {\tt PETScOptions}}
\begin{verbatim}
&PETScOptions
                                     ! Summary of Sparse Linear Solvers Available from PETSc: http://www.mcs.anl.gov/petsc/documentation/linearsolvertable.html
 KSPTYPE       = 'LGMRES'
                                     ! This parameter controls which solver is used. This is the same as petsc command line parameter -ksp_type.
                                     ! KSPTYPE = 'GMRES'
                                     ! Implements the Generalized Minimal Residual method. (Saad and Schultz, 1986) with restart
                                     ! KSPTYPE = 'LGMRES'
                                     ! Augments the standard GMRES approximation space with approximations to the error from previous restart cycles.
                                     ! KSPTYPE = 'DGMRES'
                                     ! In this implementation, the adaptive strategy allows to switch to the deflated GMRES when the stagnation occurs.
                                     ! KSPTYPE = 'PGMRES'
                                     ! Implements the Pipelined Generalized Minimal Residual method. Only PETSc 3.3
                                     ! KSPTYPE = 'KSPBCGSL'
                                     ! Implements a slight variant of the Enhanced BiCGStab(L) algorithm

 RTOL          = 1.E-20              ! the relative convergence tolerance (relative decrease in the residual norm)
 ABSTOL        = 1.E-20              ! the absolute convergence tolerance (absolute size of the residual norm)
 DTOL          = 10000.              ! the divergence tolerance
 MAXITS        = 1000                ! maximum number of iterations to use

 INITIALGUESSNONZERO = F             ! Tells the iterative solver that the initial guess is nonzero; otherwise KSP assumes the initial guess is to be zero
 GMRESPREALLOCATE    = T             ! Causes GMRES and FGMRES to preallocate all its needed work vectors at initial setup rather than the default, which is to allocate them in chunks when needed.
 SAMEPRECONDITIONER  = T             ! the preconditioner matrix is identical to that of the previous linear solve


 PCTYPE        = 'SOR'
                                     ! This parameter controls which  preconditioner is used. This is the same as petsc command line parameter -pc_type
                                     ! PCTYPE = 'SOR'
                                     ! (S)SOR (successive over relaxation, Gauss-Seidel) preconditioning
                                     ! PCTYPE = 'ASM'
                                     ! Use the (restricted) additive Schwarz method, each block is (approximately) solved with its own KSP object.
                                     ! PCTYPE = 'HYPRE'
                                     ! Allows you to use the matrix element based preconditioners in the LLNL package hypre
                                     ! PCTYPE = 'SPAI'
                                     ! Use the Sparse Approximate Inverse method of Grote and Barnard as a preconditioner
                                     ! PCTYPE = 'NONE'
                                     ! This is used when you wish to employ a nonpreconditioned Krylov method.

/
\end{verbatim}




\end{document}
