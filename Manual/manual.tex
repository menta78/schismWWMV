\documentclass[12pt]{amsart}
\usepackage{amsfonts, amsmath, latexsym, epsfig}
\usepackage{amssymb}
\usepackage{epsf}
\usepackage{url}
%\usepackage{tikz}


\newcommand{\sfA}{\ensuremath{\mathsf{A}}}
\newcommand{\RR}{\ensuremath{\mathbb{R}}}
\newcommand{\NN}{\ensuremath{\mathbb{N}}}
\newcommand{\QQ}{\ensuremath{\mathbb{Q}}}
\newcommand{\CC}{\ensuremath{\mathbb{C}}}
\newcommand{\ZZ}{\ensuremath{\mathbb{Z}}}
\newcommand{\TT}{\ensuremath{\mathbb{T}}}
\newcommand{\R}{\ensuremath{\mathbb{R}}}
\newcommand{\N}{\ensuremath{\mathbb{N}}}
\newcommand{\Q}{\ensuremath{\mathbb{Q}}}
\newcommand{\C}{\ensuremath{\mathbb{C}}}
\newcommand{\Z}{\ensuremath{\mathbb{Z}}}
\newcommand{\T}{\ensuremath{\mathbb{T}}}
\newtheorem{proposition}{Proposition}
\newtheorem{theorem}{Theorem}
\newtheorem{corollary}{Corollary}
\newtheorem{algorithm}{Algorithm}
\newtheorem{lemma}{Lemma}
\newtheorem{problem}{Problem}
\newtheorem{conjecture}{Conjecture}
\newtheorem{claim}{Claim}
\newtheorem{remark}{Remark}
\newtheorem{definition}{Definition}
\def\QuotS#1#2{\leavevmode\kern-.0em\raise.2ex\hbox{$#1$}\kern-.1em/\kern-.1em\lower.25ex\hbox{$#2$}}


\urlstyle{sf}

\DeclareMathOperator{\Aut}{Aut}
\DeclareMathOperator{\Sym}{Sym}
\DeclareMathOperator{\Isom}{Isom}
\DeclareMathOperator{\vertt}{vert}
\DeclareMathOperator{\conv}{conv}
\DeclareMathOperator{\SC}{SC}
\DeclareMathOperator{\SL}{SL}
\DeclareMathOperator{\GL}{GL}
\DeclareMathOperator{\PSL}{PSL}
\DeclareMathOperator{\Out}{Out}
\DeclareMathOperator{\Min}{Min}
\DeclareMathOperator{\Dom}{Dom}
\DeclareMathOperator{\cone}{cone}
\DeclareMathOperator{\Stab}{Stab}


\begin{document}

\author{Mathieu Dutour Sikiri\'c}
\address{Mathieu Dutour Sikiri\'c, Rudjer Boskovi\'c Institute, Bijenicka 54, 10000 Zagreb, Croatia}
\email{mdsikir@irb.hr}


\title{Short Manual of WWMIII}
\date{}

\maketitle


\section{Introduction}
WWMIII is a third generation wave model that solves the Wave Action Equation on an unstructured mesh
It provides many feature similar to SWAN, WAM, WaveWatch III.

Here we are describing how it can used.

\section{description of input file}
The input file is typically names {\tt wwminput.nml} and the program is run as
\begin{verbatim}
[mathieu@neptun ~]$ wwmadv
\end{verbatim}
But it is possible to give it other names, which are then used as
\begin{verbatim}
[mathieu@neptun ~]$ wwmadv wwminput_test.nml
\end{verbatim}
The {\tt wwminput.nml} file is using a standard NAMELIST type:
\begin{enumerate}
\item Logical tests are indicated as {\bf F/T}
\item Optional items are set to a default value.
\item Some systems are very strict with the input that they accept. All strings should be delimited by '   '. Final comma at the end 
\end{enumerate}
We will see below all aspects of it:

\subsection{Section {\tt PROC}}
\begin{verbatim}
&PROC
 PROCNAME       = 'wwm_test'         ! Project Name
 DIMMODE        = 2                  ! Mode of run (ex: 1 = 1D, 2 = 2D) always 2D when coupled to SELFE
 LSTEA          = F                  ! steady mode; under development
 LQSTEA         = F                  ! Quasi-Steady Mode; In this case WWM-II is doing subiterations defined as DELTC/NQSITER unless QSCONVI is not reached
 LSPHE          = F                  ! Spherical coordinates
 LNAUTIN        = T                  ! Nautical convention for all inputs given in degrees (suggestion: T)
                                     ! If T, 0 is _from_ north, 90 is from east etc;
                                     ! If F, maths. convention - 0: to east; 90: going to north
 LMONO_IN       = F                  ! For prescribing monochromatic wave height Hmono as a boundary conditions 
 LMONO_OUT      = F                  ! Output wave heights in terms of Lmono
 BEGTC          = '20071101.000000'  ! Time for start the simulation, ex:yyyymmdd. hhmmss
 DELTC          = 600                ! Time step (not used with SELFE)
 UNITC          = 'SEC'              ! Unity of time step
 ENDTC          = '20071102.000000'  ! Time for stop the simulation, ex:yyyymmdd. hhmmss
 DMIN           = 0.001              ! Minimum water depth. This is not used in selfe; with selfe this is set automatically to h0 in param.in
/
\end{verbatim}
The {\tt LMONO\_IN} and {\tt LMONO\_OUT} refer to the wave height that is accepted in input. If set to false then the wave height are considered to be significant wave height otherwise they are considered as monochromatic wave heights.


\subsection{Section {\tt COUPL}}
\begin{verbatim}
&COUPL
 LCPL           = T                  ! Couple with current model ... main switch - keep it on
 LROMS          = F                  ! ROMS (set as F)
 LTIMOR         = F                  ! TIMOR (set as F)
 LSHYFEM        = F                  ! SHYFEM (set as F)
 RADFLAG        = 'LON'
 LETOT          = F                  ! Option to compute the wave induced radiation stress. If .T
. the radiation stress is based on the integrated wave spectrum
                                     ! e.g. Etot = Int,0,inf;Int,0,2*pi[N(sigma,theta)]dsigma,dth
eta. If .F. the radiation stress is estimated as given in Roland et al. (2008) based
                                     ! on the directional spectra itself. It is always desirable 
to use .F., since otherwise the spectral informations are truncated and therefore
                                     ! LETOT = .T., is only for testing and developers!
 NLVT           = 10                 ! Number of vertical Layers; not used in other coupling mode
s except SELFE
 DTCOUP         = 600.               ! Couple time step - not used when coupled to SELFE
/
\end{verbatim}
The logical {\tt LCPL} determines whether the WWM is coupled or not. If True, then it can be
\begin{itemize}
\item With SELFE when compiled with SELFE.
\item With ROMS when compiled with the PGMCL library and ROMS
\item With TIMOR (using pipes) if LTIMOR=T
\item With ROMS (using pipes) if LROMS=T
\item With SHYFEM (using pipes) if LSHYFEM=T
\end{itemize}






\end{document}
